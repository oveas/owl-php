\section{exa.form.php}
This example shows how to create a form and display it. The example code in the comments can be used in the mainpage, e.g. index.php. \begin{DoxyAuthor}{Author}
Oscar van Eijk, Oveas Functionality Provider
\end{DoxyAuthor}

\begin{DoxyCodeInclude}
<?php
/*
 * This class shows how to add a login form to the document.
 * 
 * In this example, a new form is created with a username field, a password field
       and a
 * submit button.
 * A table is created with one row for each field, and the table is set as conten
      t for a fieldset.
 * Next, the fieldset is defined as the content for the form and finally a new DI
      V is created
 * which whill get the form as the content.
 *
 * To display the form, this code can be used:
 *

// First, get all required instances
$document   = OWL::factory('Document', 'ui');

// Create the main containers
$GLOBALS['MYAPP']['BodyContainer'] = new Container('div', '', array('class' => 'b
      odyContainer'));

// Get this classfile (assuming the full path is '(MYAPP_UI)/user/login.php')
if (($_lgi = OWLloader::getArea('login', MYAPP_UI . '/user') !== null) {
        // Add it to the body container
        $_lgi->addToDocument($GLOBALS['MYAPP']['BodyContainer']);
}

// Load style and add content to the document
$document->loadStyle(MYAPP_CSS . '/my-application.css');
$document->addToContent($GLOBALS['MYAPP']['BodyContainer']);

// Now display the document
echo $document->showElement();

 */

// First we must load the required Form class
if (!OWLloader::getClass('form')) {
        trigger_error('Error loading the Form class', E_USER_ERROR);
}
// In this example, a table is used, wo we must add the table class as well
if (!OWLloader::getClass('table')) {
        trigger_error('Error loading the Table class', E_USER_ERROR);
}

class LoginArea extends ContentArea
{
        /*
         * Generate the Login form and add it to the document
         */
        public function loadArea()
        {
                // Check if the current user can see this form
                if ($this->hasRight('readanonymous', OWL_ID) === false) {
                        return false; // Note; 'false' here causes OWLloader::get
      Area() to return false!
                }

                // Create a new form. The first argument defines the dispatcher, 
      second is the form name
                $_frm = new Form(
                          array(
                                 'application' => 'my-application'
                                ,'include_path' => 'MYAPP_BO'
                                ,'class_file' => 'myappuser'
                                ,'class_name' => 'myAppUser'
                                ,'method_name' => 'doLogin'
                        )
                        , array(
                                 'name' => 'loginForm'
                        )
                );

                // Start a new table that will be the placeholder for the login f
      orm
                $_liTable = new Table(array('style'=>'border: 0px; width: 100%;')
      );

                // Add a new row for the Username field
                $_rowU = $_liTable->addRow();
                // Add the Username field to the form
                $_usrFld = $_frm->addField('text', 'usr', '', array('size' => 20)
      );
                // Get the translation for the label
                $_usrLabel = $this->trn('Username');
                // Create a <label> container for the username field with the tra
      nslation as content
                $_usrContnr = new Container('label', $_usrLabel, array(), array('
      for' => &$_usrFld));
                // Add a new cell to the tablerow
                $_usrCell = $_rowU->addCell();
                // Set the <label> containter as content for the new cell
                $_usrCell->setContent($_usrContnr);
                // Add a new cell and set the form field as container
                $_rowU->addCell($_frm->showField('usr'));

                // Same story for the password
                $_rowP = $_liTable->addRow();
                $_pwdFld = $_frm->addField('password', 'pwd', '', array('size' =>
       15));
                $_pwdLabel = $this->trn('Password');
                $_pwdContnr = new Container('label', $_pwdLabel, array(), array('
      for' => &$_pwdFld));
                $_pwdCell = $_rowP->addCell();
                $_pwdCell->setContent($_pwdContnr);
                $_rowP->addCell($_frm->showField('pwd'));

                // And a bit simpler for the submit button
                $_rowS = $_liTable->addRow();
                $_frm->addField('submit', 'act', $this->trn('Login'));
                $_rowS->addCell(
                          $_frm->showField('act')
                        , array('colspan'=>2
                        , 'style'=>'text-align:center;')
                );

                // Create the fieldset containter and fill it with the table
                $_fldSet = new Container(
                          'fieldset'
                        , $_liTable->showElement()
                        , array()
                        , array('legend' => $this->trn('Login Form'))
                );

                // Add the fieldset to the form
                $_frm->addToContent($_fldSet);

                // Now create the DIV, add the form 
                $this->contentObject = new Container('div', '', array('class' => 
      'loginArea'));
                $this->contentObject->setContent($_frm);
        }
}
?>
\end{DoxyCodeInclude}
 